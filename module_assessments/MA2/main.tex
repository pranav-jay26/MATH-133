%%%%%%%%%%%%%%%%%%%%%%%%%%%%% Define Article %%%%%%%%%%%%%%%%%%%%%%%%%%%%%%%%%%
\documentclass{article}
%%%%%%%%%%%%%%%%%%%%%%%%%%%%%%%%%%%%%%%%%%%%%%%%%%%%%%%%%%%%%%%%%%%%%%%%%%%%%%%

%%%%%%%%%%%%%%%%%%%%%%%%%%%%% Using Packages %%%%%%%%%%%%%%%%%%%%%%%%%%%%%%%%%%
\usepackage{geometry}
\usepackage{graphicx}
\usepackage{amssymb}
\usepackage{amsmath}
\usepackage{amsthm}
\usepackage{empheq}
\usepackage{mdframed}
\usepackage{booktabs}
\usepackage{lipsum}
\usepackage{graphicx}
\usepackage{color}
\usepackage{psfrag}
\usepackage{pgfplots}
\usepackage{bm}
\usepackage{float}
\usepackage{listings}
%%%%%%%%%%%%%%%%%%%%%%%%%%%%%%%%%%%%%%%%%%%%%%%%%%%%%%%%%%%%%%%%%%%%%%%%%%%%%%%

% Other Settings

\lstset{
  language=R,
  basicstyle=\ttfamily\small,
  keywordstyle=\color{blue}\bfseries,
  commentstyle=\color{green!50!black},
  stringstyle=\color{orange},
  numbers=left,
  numberstyle=\tiny,
  stepnumber=1,
  numbersep=5pt,
  backgroundcolor=\color{gray!10},
  frame=single,
  breaklines=true,
  captionpos=b,
  tabsize=2
}

%%%%%%%%%%%%%%%%%%%%%%%%%% Page Setting %%%%%%%%%%%%%%%%%%%%%%%%%%%%%%%%%%%%%%%
\geometry{a4paper}

%%%%%%%%%%%%%%%%%%%%%%%%%% Define some useful colors %%%%%%%%%%%%%%%%%%%%%%%%%%
\definecolor{ocre}{RGB}{243,102,25}
\definecolor{mygray}{RGB}{243,243,244}
\definecolor{deepGreen}{RGB}{26,111,0}
\definecolor{shallowGreen}{RGB}{235,255,255}
\definecolor{deepBlue}{RGB}{61,124,222}
\definecolor{shallowBlue}{RGB}{235,249,255}
%%%%%%%%%%%%%%%%%%%%%%%%%%%%%%%%%%%%%%%%%%%%%%%%%%%%%%%%%%%%%%%%%%%%%%%%%%%%%%%

%%%%%%%%%%%%%%%%%%%%%%%%%% Define an orangebox command %%%%%%%%%%%%%%%%%%%%%%%%
\newcommand\orangebox[1]{\fcolorbox{ocre}{mygray}{\hspace{1em}#1\hspace{1em}}}
%%%%%%%%%%%%%%%%%%%%%%%%%%%%%%%%%%%%%%%%%%%%%%%%%%%%%%%%%%%%%%%%%%%%%%%%%%%%%%%

%%%%%%%%%%%%%%%%%%%%%%%%%%%% English Environments %%%%%%%%%%%%%%%%%%%%%%%%%%%%%
\newtheoremstyle{mytheoremstyle}{3pt}{3pt}{\normalfont}{0cm}{\rmfamily\bfseries}{}{1em}{{\color{black}\thmname{#1}~\thmnumber{#2}}\thmnote{\,--\,#3}}
\newtheoremstyle{myproblemstyle}{3pt}{3pt}{\normalfont}{0cm}{\rmfamily\bfseries}{}{1em}{{\color{black}\thmname{#1}~\thmnumber{#2}}\thmnote{\,--\,#3}}
\theoremstyle{mytheoremstyle}
\newmdtheoremenv[linewidth=1pt,backgroundcolor=shallowGreen,linecolor=deepGreen,leftmargin=0pt,innerleftmargin=20pt,innerrightmargin=20pt,]{theorem}{Theorem}[section]
\theoremstyle{mytheoremstyle}
\newmdtheoremenv[linewidth=1pt,backgroundcolor=shallowBlue,linecolor=deepBlue,leftmargin=0pt,innerleftmargin=20pt,innerrightmargin=20pt,]{definition}{Definition}[section]
\theoremstyle{myproblemstyle}
\newmdtheoremenv[linecolor=black,leftmargin=0pt,innerleftmargin=10pt,innerrightmargin=10pt,]{problem}{Problem}[section]
%%%%%%%%%%%%%%%%%%%%%%%%%%%%%%%%%%%%%%%%%%%%%%%%%%%%%%%%%%%%%%%%%%%%%%%%%%%%%%%

%%%%%%%%%%%%%%%%%%%%%%%%%%%%%%% Plotting Settings %%%%%%%%%%%%%%%%%%%%%%%%%%%%%
\usepgfplotslibrary{colorbrewer}
\pgfplotsset{width=8cm,compat=1.9}
%%%%%%%%%%%%%%%%%%%%%%%%%%%%%%%%%%%%%%%%%%%%%%%%%%%%%%%%%%%%%%%%%%%%%%%%%%%%%%%

%%%%%%%%%%%%%%%%%%%%%%%%%%%%%%% Title & Author %%%%%%%%%%%%%%%%%%%%%%%%%%%%%%%%
\title{Math 133: Statistical Learning Methods - Module 2 Assessment}
\author{Pranav Jayakumar}
%%%%%%%%%%%%%%%%%%%%%%%%%%%%%%%%%%%%%%%%%%%%%%%%%%%%%%%%%%%%%%%%%%%%%%%%%%%%%%%

\begin{document}
    \maketitle
    \section{Exercises}
    \vspace{0.1in}
      \subsection{Election Predictions}
      We will use voter demographic data from the 2016 United States presidential elections to predict whether or not a person voted for Donald Trump.
      \vspace{0.1in}
      \subsubsection{Linear Model}
      We will begin by fitting a linear model of the form:
      \[\hat{y} = \beta_0 + \beta_1 X_1 + \beta_2 X_2 + \beta_3 X_3 + \epsilon\]
      Where:
      \begin{enumerate}
        \item \(\hat{y}\) is \verb|trump.vote|
        \item \(\beta\) is the model coefficients
        \item \(X_1\) is \verb|non.citizen|
        \item \(X_2\) is \verb|unemployed|
        \item \(X_3\) is \verb|metro|
        \item \(\epsilon\) is the residual error
      \end{enumerate}
    
      \begin{lstlisting}
        data <- read.csv("../../data/election2016.csv")
        trump_lm <- lm(trump.vote~non.citizen+unemployed+metro, data=data)
        summary(trump_lm)
      \end{lstlisting}
      \vspace{0.1in}

      \begin{table}[H]
        \caption{Coefficients}\label{tab:coefficients}
        \vspace{0.05in}
        \centering
        \begin{tabular}{ccccc}
          \toprule
          Coefficients & Estimate & Std. Error & t-value & Pr($>|t|$) \\ 
          \midrule
          (Intercept) & 0.62064 & 0.06793 & 9.136 & \(6.65 \times 10^{-12}\) \\
          non.citizen & -1.25402 & 0.57049 & -2.198 & 0.0330 \\
          unemployed & 1.58516 & 1.15351 & 1.374 & 0.1760 \\
          metro & -0.18211 & 0.10118 & -1.800 & 0.0784 \\
          \bottomrule
        \end{tabular}
      \end{table}
      \vspace{0.1in}
      \begin{table}[H]
        \centering
        \caption{Model Performance}\label{tab:performance}
        \vspace{0.05in}
        \begin{tabular}{cc}
          \toprule
          Residual Standard Error & 0.08011 on 46 degrees of freedom \\ 
          Multiple R-squared & 0.4072 \\ 
          Adjusted R-squared & 0.3686 \\ 
          F-statistic & 10.53 on 3 and 46 DF \\ 
          p-value & \(2.157 \times 10^{-5}\)\\ 
          \bottomrule
        \end{tabular}
      \end{table}

      We observe that only the \verb|non.citizen| variable is significant ($p = 0.0330$).
      \vspace{0.1in}
      \subsubsection{Reduced Linear Model}
      We will now fit a reduced linear model of the form:
      \[\hat{y} = \beta_0 + \beta_1 X_1 + \epsilon\]
      Where:
      \begin{enumerate}
        \item \(\hat{y}\) remains as \verb|trump.vote|
        \item \(\beta\) is the model coefficients
        \item \(X_1\) is \verb|non.citizen|
        \item \(\epsilon\) remains the residual error
      \end{enumerate}
      \vspace{0.1in}
      \begin{lstlisting}
      reduced_lm <- update(trump_lm, .~.-unemployed-metro)
      summary(reduced_lm)
      \end{lstlisting}
      \vspace{0.1in}
      \begin{table}[H]
        \centering
        \caption{Reduced Model Coefficients}\label{tab:coeff_reduced}
        \vspace{0.05in}
        \begin{tabular}{ccccc}
          \toprule
          Coefficients & Estimate & Std. Error & t-value & Pr($>|t|$) \\ % Fixed line
          \midrule
          (Intercept) & 0.5979 & 0.0225 & 26.569 & \(< 2 \times 10^{-16}\) \\ 
          non.citizen & -1.9392 & 0.3783 & -5.126 & \(5.25 \times 10^{-6}\) \\
          \bottomrule
        \end{tabular}
      \end{table}
      
 \end{document}
